% -*- coding: utf-8 -*-
\listfiles
\documentclass{article}
\usepackage{fontspec}
\setmainfont{TeX Gyre Pagella}
\usepackage[UKenglish, german]{babel}
\directlua{
  local nlp = require('pdnm_nl_printer')
  local printer = nlp.new_printer('[node] ')
  local sprinter = nlp.new_simple_printer('[simple] ')
  local function cb(head)
     printer(head)
     sprinter(head)
     lang.hyphenate(head)
     return true
  end
  luatexbase.add_to_callback('hyphenate', cb, 'test_h')
%  luatexbase.add_to_callback('ligaturing', cb, 'test_l')
%  luatexbase.add_to_callback(kerning, cb, 'test_k')
%  luatexbase.add_to_callback('pre_linebreak_filter', cb, 'test_pre')
%  luatexbase.add_to_callback('post_linebreak_filter', cb, 'test_post')
}
\begin{document}
\hyphenation{
  ba{k-}{k}{ck}en
}
%Ein einfaches Beispiel.  Und noch eines.

%drucken dru"cken

\discretionary{a-}{b}{c}

%dru\discretionary{k-}{}{c}ken dru\discretionary{k-}{k}{ck}en

%pflücken

%Auflage Auf"|lage

%Brennessel Bre"nnessel

%offen finden Offizier

%finden Treff
%finden Treff

%Trick
%(keinem)
%Achtungs{\selectlanguage{UKenglish}e}rfolge

\showhyphens{
%  Auflage
%  Auf"|lage
%  backen
%  laufen
%  lau-fen
%  lau\-fen
  finden
  finden
}
\end{document}
