% -*- coding: utf-8 -*-
\listfiles
\documentclass{article}
\usepackage[utf8]{luainputenc}
\usepackage{fontspec}
\setmainfont{Latin Modern Roman}
\setmonofont{Latin Modern Mono Prop}
\usepackage[main=english, german]{babel}
\usepackage{pdnm_break-ligatures}
\begin{document}

\section{Pattern driven ligature handling in \TeX}

This is an example application of the padrinoma package demonstrating
selective, pattern driven ligature handling using dedicated word joint
patterns.  No explicit ligature handling mark-up has been used in the
examples shown below.

Example paragraphs contain a selection of fifty (roughly) alternating
words with letters that should or should not be subject to ligation.
Words are read from files \texttt{words.*} and change at every
compilation of the \texttt{.tex} source file.  Patterns can be found in
file \texttt{examples/patterns/hyph-de-1901-joint.pat.txt}.

\subsection{German examples (traditional orthography)}

% Read example word lists from files.
\bgroup
% Characters # and % are used in Lua code.
\catcode`\#=12
\catcode`\%=12
\directlua{
  local function read_words(basename, t)
     local fi = assert(io.open(basename .. '.lig', 'r'))
     local file = fi:read('*a')
     fi:close()
     t.lig = {}
     for word in unicode.utf8.gmatch(file, '(%w+)') do
        table.insert(t.lig, word)
     end
     local fi = assert(io.open(basename .. '.nolig', 'r'))
     local file = fi:read('*a')
     fi:close()
     t.nolig = {}
     for word in unicode.utf8.gmatch(file, '(%w+)') do
        table.insert(t.nolig, word)
     end
  end

  words = {
     ff = {},
     fi = {},
     fl = {},
     ffi = {},
     ffl = {},
     st = {},
     tz = {},
  }

  read_words('words.german.ff', words.ff)
  read_words('words.german.fi', words.fi)
  read_words('words.german.fl', words.fl)
  read_words('words.german.ffi', words.ffi)
  read_words('words.german.ffl', words.ffl)
  read_words('words.german.st', words.st)
  read_words('words.german.tz', words.tz)

  function paragraphs(words, num_p, num_w)
      for i = 1,num_p do
         for w = 1,num_w / 2 do
            tex.print(words.lig[math.random(#words.lig)] .. ' ' .. words.nolig[math.random(#words.nolig)] .. ' ')
         end
         tex.print('\par')
      end
   end
   math.randomseed(os.time())
}
\egroup

\subsubsection{Letter combination f + f}
\begin{otherlanguage}{german}
\directlua{paragraphs(words.ff, 1, 50)}
\end{otherlanguage}

\subsubsection{Letter combination f + i}
\begin{otherlanguage}{german}
\directlua{paragraphs(words.fi, 1, 50)}
\end{otherlanguage}

\subsubsection{Letter combination f + l}
\begin{otherlanguage}{german}
\directlua{paragraphs(words.fl, 1, 50)}
\end{otherlanguage}

\subsubsection{Letter combination f + f + i}
\begin{otherlanguage}{german}
\directlua{paragraphs(words.ffi, 1, 50)}
\end{otherlanguage}

\subsubsection{Letter combination f + f + l}
\begin{otherlanguage}{german}
\directlua{paragraphs(words.ffl, 1, 50)}
\end{otherlanguage}

\subsubsection{Letter combination s + t}
\begin{otherlanguage}{german}
\setmainfont[Ligatures=Historic]{Linux Libertine O}
\directlua{paragraphs(words.st, 1, 50)}
\end{otherlanguage}

\subsubsection{Letter combination t + z}
\begin{otherlanguage}{german}
\setmainfont[Ligatures=Rare]{Linux Libertine O}
\directlua{paragraphs(words.tz, 1, 50)}
\end{otherlanguage}

\end{document}
