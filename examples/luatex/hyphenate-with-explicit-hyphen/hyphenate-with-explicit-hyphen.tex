% -*- coding: utf-8 -*-
\listfiles
\documentclass{article}
\usepackage{fontspec}
\usepackage{multicol}
\usepackage[main=english, ngerman]{babel}
\usepackage{padrinoma}
\PadrinomaRegisterManipulation{
  language = ngerman,
  patterns = ../../patterns/hyph-de-1996-compound.pat.txt,
  module = pdnm_hyphenate-with-explicit-hyphen,
  id = explicit-hyphen,
}
\begin{document}
This is an example application of the padrinoma package illustrating
automatic hyphenation of words containing a hyphen with Lua\TeX.  No
explicit hyphenation mark-up has been used in the source document.

In this example, word particles are hyphenated independently using
dedicated German compound word hyphenation patterns, read from file
\begin{center}
\verb+examples/patterns/hyph-de-1996-compound.pat.txt+
\end{center}
with inner word left and right hyphenation minima set to a value of~4.
If there are no such patterns available for your language, regular
hyphenation patterns can be used just as well.  But it is recommended to
set inner word hyphenation minima to a large value, \emph{e.g.},~8, in
that case.  Currently, these values are hard-coded, but they can be
adjusted in file
\begin{center}
\verb+pdnm_hyphenate-with_explicit-hyphen.lua+
\end{center}

See \texttt{log} file for a list of hyphenated German words containing
explicit hyphens.  Hyphenation results can be found after the colon in
lines starting \verb+[pdnm]+.  Hyphens in input are replaced by an equal
sign in output.  In the example \emph{Notaus-Schalter}, if inner word
hyphenation minima were set to smaller values, that would have been
hyphenated \emph{Not-aus=Schalter} as \emph{Notaus} itself is a compound
word and recognized as such by the given patterns.

\begin{otherlanguage}{ngerman}
\showhyphens{
  Autobahn
  Notaus-Schalter
  Arbeiter-Unfallversicherung
  Zwei-Drittel-Abstimmungsmehrheit
}
\end{otherlanguage}
\end{document}
