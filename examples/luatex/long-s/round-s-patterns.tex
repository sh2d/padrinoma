% -*- coding: utf-8 -*-
\listfiles
\documentclass{article}
\usepackage{fontspec}
\setmonofont{Latin Modern Mono Prop}
\usepackage[main=english, german]{babel}
\usepackage{pdnm_round-s-patterns}
\begin{document}

\section{Black-letter type and \TeX}

This is an example application of the padrinoma package demonstrating
automatic, context-aware \emph{round-s} to \emph{long-s} conversion
(\emph{s}~$\rightarrow$~\emph{ſ}).  Words in input using \emph{round-s}
form exclusively, are automatically converted to a mixture
of \emph{round-} and \emph{long-s} forms in output according to German
typographic conventions.  No explicit mark-up has been used in this
document.

This feature is targetting at typesetting with black-letter fonts.  Due
to missing good quality black-letter OpenType fonts in TeX Live, example
text below is typeset using Latin Modern Roman font, though.  The reader
is encouraged to change the font in the source document to his favourite
one.

This document uses patterns indicating occurrences of \emph{round-s}
glyphs.  Patterns can be found in
file \texttt{examples/patterns/gsub-long-s-neg-de-1901.pat.txt}.


\subsection{German example (traditional orthography)}

% Read example word lists from files.
\bgroup
% Characters # and % are used in Lua code.
\catcode`\#=12
\catcode`\%=12
\directlua{
  local function read_words(basename, t)
     local fi = assert(io.open(basename .. '.rounds', 'r'))
     local file = fi:read('*a')
     fi:close()
     t.rounds = {}
     for word in unicode.utf8.gmatch(file, '(%w+)') do
        table.insert(t.rounds, word)
     end
     local fi = assert(io.open(basename .. '.longs', 'r'))
     local file = fi:read('*a')
     fi:close()
     t.longs = {}
     for word in unicode.utf8.gmatch(file, '(%w+)') do
        table.insert(t.longs, word)
     end
  end

  words = {
     s = {},
  }

  read_words('words.german.s', words.s)

  function paragraphs(words, num_p, num_w)
      for i = 1,num_p do
         for w = 1,num_w / 2 do
            tex.print(words.rounds[math.random(#words.rounds)] .. ' ' .. words.longs[math.random(#words.longs)] .. ' ')
         end
         tex.print('\par')
      end
   end
   math.randomseed(os.time())
}
\egroup

\begin{otherlanguage}{german}
Dresden Dresdner Oskar obskur grotesk Imbiss lass ausser missverstanden
Missgeschick Schiessbude

\directlua{paragraphs(words.s, 1, 50)}
\end{otherlanguage}

\end{document}
