% -*- coding: utf-8 -*-
\listfiles
\documentclass{article}
\usepackage{fontspec}
\usepackage{multicol}
\usepackage[main=german, english]{babel}
\directlua{
  local nlm = require('pdnm_nl_manipulation')
% Parameters are:
%  * name of a pattern file (full file name),
%  * name of a Lua module implementing a manipulation (base file name),
%  * a unique ID for the manipulation (an arbitrary, non-nil Lua value).
  nlm.register_manipulation('nstd-hyph-de-1901.pat.txt', 'pdnm_nstd-hyph-de-1901', 'nstd-hyph-de-1901')
}
\begin{document}

\begin{otherlanguage*}{english}
  This document contains some paragraphs of random German words and
  named entities where non-standard hyphenation may or may not be
  desirable.  In German, letters \emph{ck} turn into \emph{k-k} when
  hyphenated.  This is achieved here by replacing a node representing
  the letter \emph{c} by a discretionary node of the form
  \verb+\discretionary{k-}{}{c}+.  The replacement is only applied where
  indicated by suitable non-standard hyphenation patterns.  That way, no
  physical mark-up is required in a text to enable \emph{ck}
  hyphenation.  Have a look at the \verb+log+ file for a list of fully
  hyphenated words.
\end{otherlanguage*}

\begin{multicols}{5}
% Make hyphenation desirable.
\hyphenpenalty=-100
\doublehyphendemerits=-100
\finalhyphendemerits=-100
% # is used in Lua code.
\catcode`\#=12
\directlua{
   words = {
      % random words
      'Acker', 'Ackerböschung', 'Attacke', 'auflockern',
      'backen', 'Backstube', 'Bestecke', 'Bestecks', 'blicken',
      'Blockade', 'Brücke',
      'Cricket', 'Kricket',
      'Dackel', 'Deckel', 'Dickicht', 'dreckig', 'dreieckig', 'drucken',
      'Ecke', 'eckig', 'entzückend', 'entzückt', 'erschrickst',
      'flackern', 'Fleck', 'fleckige', 'flicken', 'Flickschuster',
      'Flocken',
      'Gegacker', 'Glöckchen', 'Glocke', 'gluckern',
      'Hacken', 'Hecke', 'Heckmeck',
      'Hockey',
      'Jockey',
      'Lackschaden', 'Leck', 'lecker', 'lockig', 'Lücke',
      'Macke', 'Meckerer', 'Mücke',
      'packen', 'packst', 'packt', 'pflücken', 'Pickel', 'Pickelhaube',
      'Pickhacke', 'Picknickkorb', 'Pocken', 'puckern',
      'recken', 'reckst', 'Reckstange', 'Röcke', 'Rucksack', 'Rücken',
      'Rückerstattung', 'Rücklicht', 'Rückzug',
      'Sack', 'schlecken', 'schmecke', 'Schnecke', 'Schockwelle',
      'Socken', 'Steckdose', 'stecken', 'Stickoxid', 'Stöcker',
      'Stockfisch', 'Stockwerk', 'strecken', 'streckst', 'Strecksprung',
      'Streckung', 'Stuckateur', 'Stücke', 'Stücken', 'Stückchen',
      'stückweise',
      'Tischdecke', 'trocknen', 'trockne', 'trocknest', 'Trockner',
      'verstecken', 'Versteckspiel', 'vertrackt',
      'Weckruf', 'Wicke', 'Wicklung',
      'zickig', 'Zickzack', 'Zucker', 'Zuckerbäcker', 'zweckmäßig',
      'zwicke', 'zwickst',
      % systematic letter combinations
      'Blockade', 'Deckadresse',%cka
      'Drucker', 'Abdruckerlaubnis',%cke
      'lockig', 'Schmuckindustrie',%cki
      'Backofen',%cko
      'Entdeckung', 'Druckunterschied',%cku
      'Strickjacke',%ckj
      'Druckänderung',%ckä
      'Backöfen',%ckö
      'Rückübertragung',%ckü
      % named entities
      'Ahlbeck', 'Ahlbecker', 'Ahlbecks',
      'Auckland',
      'Bad Säckingen',
      'Bad Wilsnack', 'Bad Wilsnacker', 'Bad Wilsnacks',
      'Barack', 'Baracks',
      'Becker', 'Beckham', 'Beckmann', 'Becks', 'Becky',
      'Bismarck', 'Bismarcks', 'Bismarcksche',
      'Blacky',
      'Borckward', 'Borckwart', 'Borckwardt',
      'Brinckmann',
      'Brockdorf', 'Brockdorff', 'Brockhoff',
      'Bruckmüller',
      'Buckow',
      'Deckert',
      'Delbrück', 'Delbrücks',
      'Dick', 'Dicks', 'Dickens',
      'Dircksen',
      'Eckart', 'Eckehard', 'Eckehardt', 'Eckhard', 'Eckhardt',
      'Eckernförde',
      'Eickstedt',
      'Flecken Zechlin',
      'Fock', 'Focks', 'Focke',
      'Franck', 'Francks', 'Francke', 'Franckes', 'Francken',
      'Frederick', 'Fredericks',
      'Fricke', 'Fricktal',
      'Gauck', 'Gaucks',
      'Gecko',
      'Glienicke',
      'Haack', 'Haacks',
      'Hancock', 'Hancocks',
      'Hausruckviertel',
      'Hendricks',
      'Hickory',
      'Hockenheim',
      'Huckleberry',
      'Hunsrück', 'Hunsrücks',
      'Innsbruck', 'Innsbrucker', 'Innsbrucks',
      'Jack', 'Jacks', 'Jackson', 'Jacksonville',
      'Kentucky',
      'Klickermann',
      'Knickerbocker',
      'Kuckuck', 'Kuckucks',
      'Lamarck', 'Lamarcks',
      'Lübeck', 'Lübecker', 'Lübecks', 'Lübbecke',
      'Luckenwalde', 'Luckner',
      'Mackensen',
      'Mackenzie', 'McKenzie',
      'McKean',
      'Mackie Messer',
      'Mecklenburg',
      'Mischnick', 'Mischnicks',
      'Mount McKinley',
      'Neckar', 'Neckarsulm',
      'Nick', 'Nicks', 'Nicki', 'Nicky',
      'Noack', 'Noacks',
      'ocker',
      'Osnabrück', 'Osnabrücker', 'Osnabrücks',
      'Packard',
      'Patrick', 'Patricks',
      'Peck', 'Pecks',
      'Planck', 'Plancks', 'Plancksche',
      'Pückler',
      'Rebecka',
      'Recklinghausen', 'Recknagel',
      'Rockefeller', 'Rockford', 'Rocky Mountains',
      'Rostock', 'Rostocker', 'Rostocks',
      'Saarbrücken',
      'Schenck',
      'Schmöckwitz',
      'Schreckenberg',
      'Schweickhardt',
      'Senckenberg',
      'Sickingen',
      'Steckborn',
      'Steinbrück', 'Steinbrücks',
      'Stockerau', 'Stockhausen', 'Stockholm',
      'Struck', 'Strucks', 'Strucksche',
      'Stücklen',
      'Tresckow',
      'Uckermark', 'Ueckermünde',
      'Vicki', 'Vicky',
      'Vöcklabruck', 'Vöcklabrucks',
      'Weitzsäcker', 'Weizsäcker',
      'Westrick', 'Westricks',
      'Winckelmann',
      'Yorck', 'Yorcks', 'Yorckscher',
      'Zweibrücken',
      'Zwickau',
   }
   math.randomseed(os.time())
   for i = 1,3 do
      for w = 1,90 do
         tex.print(words[math.random(#words)] .. ' ')
      end
      tex.print('\par')
   end
}
\end{multicols}
% Output a list of hyphenated words to log file.
\showhyphens{
  \directlua{
     for _,word in ipairs(words) do
        tex.print(word .. ' ')
     end
  }
}
\end{document}
