% -*- coding: utf-8 -*-
\listfiles
\documentclass{article}
\usepackage{fontspec}
\usepackage{multicol}
\usepackage[main=english, german]{babel}
\usepackage{padrinoma}
% Parameters are:
%  * a TeX language identifier,
%  * name of a pattern file (full file name),
%  * name of a Lua module implementing a manipulation (base file name),
%  * a unique ID for the manipulation (interpreted as a Lua string).
\PadrinomaRegisterManipulation{german}{nstd-hyph-de-1901.pat.txt}{pdnm_nstd-hyph-de-1901}{nstd-hyph-de-1901}
\begin{document}

\section{Automatic non-standard hyphenation with Lua\TeX}

This is an example application of the padrinoma package illustrating
automatic German non-standard hyphenation with Lua\TeX.  No explicit
non-standard hyphenation mark-up has been used in the source document.

\subsection{German non-standard hyphenation rules}

Besides standard hyphenation, in traditional German orthography, there
are two non-standard hyphenation rules that call for more complex
operations than the mere insertion of a hyphen at the end of a line:

\begin{description}

\item[ck-rule] When hyphenated, letters \emph{ck} turn into
  \emph{k-k}:\par
\begin{tabular*}{\linewidth}{l@{\extracolsep{\fill}}r}
  \emph{Dackel} vs. \emph{Dak-kel} & (dachshund)\\
\end{tabular*}

\item [triple consonant rule] In compound words, of three equal adjacent
  consonants followed by a vocal, one consonant is cancelled out.  As an
  example, the words \emph{Schiff} and \emph{Fahrt} can be combined to
  the word \emph{Schiffahrt}.  The second part of this rule says that
  during hyphenation, cancelled consonants reappear:\par
  \begin{tabular*}{\linewidth}{l@{\extracolsep{\fill}}r}
    \emph{Schiffahrt} vs. \emph{Schiff-fahrt} & (shipping)\\
  \end{tabular*}

\end{description}
But note, not every \emph{ck} is subject to hyphenation and the pattern
‘word boundary with double consonant followed by a vocal’ indeed rarely
stems from application of the three consonant rule:\strut\par
\noindent\begin{tabular*}{\linewidth}{l@{\extracolsep{\fill}}r}
  \emph{Steck-dose}, not Stek-kdose & (socket)\\
  \emph{Rohstoff-industrie}, not Rohstoff-findustrie & (extractive industry)\\
\end{tabular*}

\subsection{Handling non-standard hyphenation manually}

Traditional \TeX\ provides the \verb+\discretionary+ command for manual
handling of non-standard hyphenation.  Using this command, the German
word \emph{Dackel} can be typed-in as
\verb+Da\discretionary{k-}{}{c}kel+.  The Babel and Polyglossia packages
provide short-cuts so that one can also type \verb+Da"ckel+.  But the
disadvantages of these solutions are that, in languages demanding
non-standard hyphenation, authors need to manually apply non-standard
hyphenation mark-up throughout a text, which makes a source document
look cluttered and---more important---application of the mark-up demands
mental attention and distracts from the actual task of writing.

With Lua\TeX, the situation improved.  Hyphenation exceptions may now
contain \verb+\discretionary+ commands, like
\verb+\hyphenation{Da{k-}{}{c}kel}+, so that non-standard hyphenation
mark-up isn't needed anymore for the words handled in the preamble of a
document.  But still, that way, only words maintained in an explicit
list become subject to non-standard hyphenation.

\subsection{Automatic non-standard hyphenation}

The \texttt{padrinoma} package is an attempt to solve the
above-mentioned problems.  The following sections contain a selection of
German words and named entities (changing at every compilation run)
where non-standard hyphenation may or may not be desirable and is
applied automatically.  The document source contains neither mark-up nor
hyphenation exceptions.  Discretionaries are inserted fully
automatically at Lua\TeX's node level at positions indicated by
dedicated non-standard hyphenation patterns.  Patterns are read from
file \verb+pdnm_nstd-hyph-de-1901.pat.txt+.  A list of fully hyphenated
words can be found in the \verb+log+ file.

For debugging purposes, a list of all words with non-standard
hyphenation patterns only applied is written to a file with the name of
the file patterns were read from augmented by the extension
\verb+.spots+.  A hyphen indicates where non-standard hyphenation is to
be applied within a word, like \verb+Dac-kel+ or \verb+Schif-fahrt+.
Wrong matches, such as \verb+Rebec-ka+ (non-standard hyphenation is
discouraged within names) or \verb+Schlep-panker+, are due to still
imperfect non-standard hyphenation patterns.  Patterns are subject to
further improvements.  If you find examples of wrong German non-standard
hyphenation with the current pattern set, please send them to
\verb+trennmuster@dante.de+.

% # is used in Lua code.
\catcode`\#=12
\directlua{
   words_ck = {
      % random words
      'Acker', 'Ackerböschung', 'Attacke', 'auflockern',
      'backen', 'Backstube', 'Bestecke', 'Bestecks', 'blicken',
      'Cricket', 'Kricket',
      'Dackel', 'Deckel', 'Dickicht', 'dreckig', 'dreieckig', 'drucken',
      'Ecke', 'eckig', 'entzückend', 'entzückt', 'erschrickst',
      'flackern', 'Fleck', 'fleckige', 'flicken', 'Flickschuster',
      'Flocken',
      'Gegacker', 'Glöckchen', 'Glocke', 'gluckern',
      'Hacken', 'Hecke', 'Heckmeck',
      'Hockey',
      'Jockey',
      'Lackschaden', 'Leck', 'lecker', 'lockig', 'Lücke',
      'packen', 'packst', 'packt', 'pflücken', 'Pickel', 'Pickelhaube',
      'Pickhacke', 'Picknickkorb',
      'recken', 'reckst', 'Reckstange', 'Röcke', 'Rucksack', 'Rücken',
      'Rückerstattung',
      'Sack', 'schlecken', 'schmecke', 'Schnecke', 'Schockwelle',
      'Socken', 'Steckdose', 'stecken',
      'Stockfisch', 'Stockwerk', 'strecken', 'streckst', 'Strecksprung',
      'Streckung', 'Stuckateur', 'Stücke', 'Stücken', 'Stückchen',
      'stückweise',
      'Weckruf', 'Wicke', 'Wicklung',
      'zickig', 'Zickzack', 'Zucker', 'zweckmäßig',
      'Zuckerbäcker', 'Zucker-Bäcker',
      % systematic letter combinations
      'Blockade', 'Deckadresse',%cka
      'Drucker', 'Abdruckerlaubnis',%cke
      'lockig', 'Schmuckindustrie',%cki
      'Backofen',%cko
      'Entdeckung', 'Druckunterschied',%cku
      'Strickjacke',%ckj
      'Druckänderung',%ckä
      'Backöfen',%ckö
      'Rückübertragung',%ckü
      % named entities
      'Deckert',
      'Eckart', 'Eckehard', 'Eckehardt', 'Eckhard', 'Eckhardt',
      'Eckernförde',
      'Fricke', 'Fricktal',
      'Glienicke',
      'Hendricks',
      'Hockenheim',
      'Huckleberry',
      'Innsbruck', 'Innsbrucker', 'Innsbrucks',
      'Knickerbocker',
      'Kuckuck', 'Kuckucks',
      'Lübeck', 'Lübecker', 'Lübecks', 'Lübbecke',
      'Luckenwalde', 'Luckner',
      'Mackie Messer',
      'Mecklenburg',
      'Mount McKinley',
      'Neckar', 'Neckarsulm',
      'Osnabrück', 'Osnabrücker', 'Osnabrücks',
      'Packard',
      'Pückler',
      'Rebecka',
      'Recklinghausen', 'Recknagel',
      'Rockefeller', 'Rockford', 'Rocky Mountains',
      'Rostock', 'Rostocker', 'Rostocks',
      'Saarbrücken',
      'Schmöckwitz',
      'Schreckenberg',
      'Schweickhardt',
      'Senckenberg',
      'Sickingen',
      'Stockerau', 'Stockhausen', 'Stockholm',
      'Tresckow',
      'Uckermark', 'Ueckermünde',
      'Weitzsäcker', 'Weizsäcker',
      'Winckelmann',
      'Yorck', 'Yorcks', 'Yorckscher',
      'Zweibrücken',
      'Zwickau',
    }
    words_triple = {
      % triple consonant words
      'Baustoffabrik',
      'Schiffahrt',
      'Kunststoffasern',
      'Kunststoffenster',
      'Baustoffirma',
      'Zellstoffirma',
      'schifförmig',
      'Geröllawine',
      'Metallegierung',
      'Abfülleistung',
      'Nullösung',
      'Dämmaßnahme',
      'Klemmechanismus',
      'Kammolch',
      'Brennessel', 'Brennessel-Tee',
      'Kennummer',
      'Kreppapier',
      'Kippunkt',
      'Ballettanz',
      'Schrottank',
      'Fetteilchens',
      'Schnittiefe',
      % non-triple consonant words
      'Baustoffindustrie',
      'Brotteig',
      'Knettiere',
      'Kunststoffascher',
      'Kunststoffindustrie',
      'Kunststofflasche',
      'Schleppanker',
      'Schrottanker',
      'Bettruhe',
   }
   function paragraphs(words, num_p, num_w)
      for i = 1,num_p do
         for w = 1,num_w do
            tex.print(words[math.random(#words)] .. ' ')
         end
         tex.print('\par')
      end
   end
   math.randomseed(os.time())
}

\subsection{Non-standard hyphenation examples}

\begin{multicols}{5}[\subsubsection{\emph{ck} rule}]
\selectlanguage{german}
% Make hyphenation desirable.
\hyphenpenalty=-100
\doublehyphendemerits=-100
\finalhyphendemerits=-100
\directlua{
  paragraphs(words_ck, 2, 40)
}
\end{multicols}

\begin{multicols}{2}[\subsubsection{Triple consonant rule}]
\selectlanguage{german}
% Make hyphenation desirable.
\hyphenpenalty=-100
\doublehyphendemerits=-100
\finalhyphendemerits=-100
\directlua{
  paragraphs(words_triple, 2, 50)
}
\end{multicols}

% Output a list of hyphenated words to log file.
\begin{otherlanguage}{german}
\showhyphens{
  \directlua{
     for _,word in ipairs(words_ck) do
        tex.print(word .. ' ')
     end
     for _,word in ipairs(words_triple) do
        tex.print(word .. ' ')
     end
  }
}
\end{otherlanguage}
\end{document}
